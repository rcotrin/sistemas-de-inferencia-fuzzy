% configuração da fonte, pode ser courier ou times

\documentclass[courier]{uninove-ppgi}

% local definitions (utilizado para controlar a posição do número da equação)

\newcommand{\numberequation}[1]{\addtocounter{equation}{#1}\tag{\theequation}}

\begin{document}

% parametros de capa e folha de rosto (é necessário configurar todos)

\Universidade{UNIVERSIDADE NOVE DE JULHO - UNINOVE}
\Programa{PROGRAMA DE PÓS GRADUAÇÃO EM INFORMÁTICA E GESTÃO DO CONHECIMENTO - PPGI} 	
\Autor{GUILHERME-RAFAEL-SAULO} 
\Titulo{VALIDAR ARTIGO CIENTIFICO BASEADO EM UM SISTEMA FUZZY UTILIZANDO PROGRAMAÇÃO PRÓPRIA}
\Tipoexame{Trabalho cientifico}
\Titulacao{Mestre}
\Orientador{Professor e Dr. Cleber}
\Ano{2021}


% gera a capa

\capa


% gera folha de rosto

\folharosto

		   
% configurações do resumo em portugues

\PalavrasChave{Sistema Fuzzy, Sistema Inteligente, Validar fuzzy, XXXXXXX, XXXXXXXXX.}

\begin{resumo}

Atendento a solicitação de criação de trabalho junto a instituição Universidade Nove de Julho, para a matéria de Sistemas de Inferência Fuzzy (SIF),
tendo como Professor responsável Prof. DR. Cleber. O proposto foi, realizar uma pesquisa afim de indentificar um artigo para que fosse simulado toda a estrutura de um SIF, 
sendo essa criação desenvolvida utilizando novo sistema, ou seja, sem utilização de softwares ou plugins prontos de mercado, após essa elaboração analisar
as saídas do modelo de MANDAQUI com saídas de um sistema por Takagi-Sugeno, ou o inverso, identificando quais foram as melhores saídas olhando para o universo de 
discurso e varíáveis linguísticas escolhidas para o trabalho.

\end{resumo}


% configurações do resumo em inglês

\KeyWords{Fuzzy System, Smart System, Validate fuzzy, XXXXXXX, XXXXXXXXX.}

\begin{abstract}

I attend to the request to create work with the institution Universidade Nove de Julho, for the subject of Fuzzy Inference Systems (SIF),
having as responsible Professor Prof. DR. Cleber. The proposal was to carry out a search in order to identify an article so that the entire structure of a SIF could be simulated,
this creation being developed using a new system, that is, without the use of ready-to-market software or plugins, after this elaboration, analyze
the outputs of the MANDAQUI model with outputs from a system by Takagi-Sugeno, or the reverse, identifying which were the best outputs looking at the universe of
speech and language variables chosen for the work.

\end{abstract}

% Sumário

\tableofcontents 
\thispagestyle{empty}
		   
% Lista de figuras

\listoffigures
\thispagestyle{empty}

% Lista de algoritmos

\listofalgorithms
\thispagestyle{empty}

% lista de abreviaturas (não colocar espaçamentos adicionais, pois eles contam dentro do environment)

\begin{listaabreviaturas}%
	MM & Morfologia matemática \\
	CC & Componente conexo \\
	EE & Elemento estruturante \\
	MS & Mumford-Shah \\
	\textit{poset} &  Acrônimo para a expressão em inglês \textit{partially ordered set}\\
				   &  (em português: conjunto parcialmente ordenado) \\
	\textit{pixel} &  Acrônimo para a expressão em inglês \textit{picture element}\\
				   &  (em português: elemento da imagem)
\end{listaabreviaturas}

% lista de símbolos (não colocar espaçamentos adicionais, pois eles contam dentro do environment)
% atualmente a lista não suporta multiplas páginas, ou seja ela quebra a lista inteira.

\begin{listasimbolos}%
	\simbolos{Conceitos básicos} {%		
		$ \mathbb{Z} $ & Conjunto dos números inteiros \\						
		$ \mathbb{N} $ & Conjunto dos números naturais \\		
		$ \mathbb{R}^+ $ & Conjunto	dos números reais positivos \\	 		
	}
	\simbolos{Imagens} {%	
		$ f $ & Váriavel que representa uma imagem \\			
		$ \mathcal{D} $ & Conjunto que representa o domínio da imagem \\					
		$ \mathbb{K} $ & Conjunto que representa o contradomínio da imagem \\		
	}
\end{listasimbolos}

% Corpo do documento

\chapter{Identificação do artigo}

\begin{resumocapitulo}
Foi realizado pesquisa da base de dados da Web Of Science para tentarmos identificar um artigo que fosse aderente ao proposto nesse trabalho.
Identificamos um artigo que ficou mais adequado pois este possui todas as partes de um SIF, fuzzificação, regras, Inferência e defuzzificação.
Escolhemos a variável XPTO para separar a saída na defuzzificação na variável consequente, para posteriomente comparar com o artigo e com o cálculo
em Takagi-Sugeno, para assim analisarmos qual foi a solução mais adequada.
\end{resumocapitulo}

\section{O artigo}

\subsection{Pontos do artigo}

Alguns comandos matemáticos também estão disponíveis, pode-se criar definições, proposições e provas:

\begin{definicao}{Média aritmética}
Para uma amostra $ X=\{x_1,, x_2, \ldots,x_n\} $ de observações, onde $ n $ é o número de observações, se define a média aritmética da seguinte forma:
\begin{equation}
\mu(X)=\dfrac{1}{n}\sum\limits_{x \in X}x
\end{equation}
\end{definicao}
\begin{proposicao}
Se $ k $ é uma constante então multiplicar a média de uma amostra $ X $ é o mesmo de multiplicar cada elemento de $ X $ por $ k $, isto é, $ k \times \mu(X) = \dfrac{1}{n} \sum\limits_{x \in X}x\times k $.
\end{proposicao}
\begin{prova}
Desenvolve-se a igualdade:
\begin{align*}
k \times \mu(X) &= \dfrac{1}{n} \sum\limits_{x \in X}xk \\
& \Longleftrightarrow  \dfrac{(x_1k,x_2k, \ldots, x_nk)}{n} \\
& \Longleftrightarrow  \dfrac{nk \times (x_1,x_2, \ldots, x_n)}{n} \\
& \Longleftrightarrow   k \times \dfrac{(x_1,x_2, \ldots, x_n)}{n} \\
& \Longleftrightarrow   k \times \mu(X) \numberequation{1}
\end{align*}
\end{prova}
Assim, concluí-se que $ k \times \mu(X) = \dfrac{1}{n} \sum\limits_{x \in X}x\times k $. $ \square $
 
\subsubsection{Exemplo de subsubseção}

Figuras também estão configuradas pela norma ABNT, a legenda é centralizada e a fonte da figura é recuada a esquerda:

\begin{figure}[ht!]

	\begin{center}
	
		\includegraphics[scale=0.4]{leveling1}
	
	\end{center}
	
	\caption{Uma imagem.}
	
	\fonte{\citeonline{alves:article:2017} (Adaptado pelo autor)}
	
\end{figure}

As citações podem ser feitas de duas formas: {\color{red}$\backslash$citeonline}\{chave da citação\} = \citeonline{seymor:book:1971} e {\color{red}$\backslash$cite}\{chave da citação\} = \cite{seymor:book:1971}. Note que, nas referências bibliográficas o título está em negrito, de acordo com a norma ABNT 6023, para este efeito é necessário incluir a entrada no arquivo bibtex (refs.bib).

Exemplo simples de pseudocódigo utilizando o pacote {\color{red}algorithm2e} configurado para língua portuguesa:

% controla identação
\begin{algorithm}[H]
\SetAlgoLined
\Entrada{$S,\eta, U$} 
\Saida{Número esperado de nós atingidos}
\Inicio{
	$\sigma(S) = 0$ \\
\Para{cada $u \in S$}{
	$\sigma(S)\leftarrow \sigma(S)+\textsc{Backtrack}(u,\eta,W,U)$\\
}
}
\Retorna{$\sigma(S)$}
\label{alg1}
\caption{\textsc{Esperança}}
\end{algorithm}

% Incluindo bibliografia

\bibliography{refs}
		   
\end{document}
